%% V1.0
%% by Christopher Leith, udacl@cielsystems.com
%% This is a template for Udacity projects using IEEEtran.cls

\documentclass[10pt,journal,compsoc]{IEEEtran}

\usepackage[pdftex]{graphicx}    
\usepackage{cite}
\hyphenation{op-tical net-works semi-conduc-tor}

\begin{document}

\title{WhereAmI - Robotic Localization}

\author{Christopher Leith}

\markboth{WhereAmI, Localization, Udacity}%
{}
\IEEEtitleabstractindextext{%

\begin{abstract}
In this project we create a simulated robot and demonstrate Simultaneous Localization and Mapping (SLAM) within the ROS framework to permit navigation within a small simulated "Gazebo" world. A differential drive robot is designed to operate within the Gazebo simulation and uses the ROS navigation stack and the RTAB-map-ros package for creating a map on the fly and localizing itself within that map. . The tuning of configuration parameters of the simulation, navigation stack and AMCL package neccesary for success are discussed.\end{abstract}

% Note that keywords are not normally used for peerreview papers.
\begin{IEEEkeywords}
Mobile Robot, Localization, Monte Carlo Localiztion, ROS.
\end{IEEEkeywords}}


\maketitle
\IEEEdisplaynontitleabstractindextext
\IEEEpeerreviewmaketitle
\label{sec:Introduction}
\section{Introduction}
\IEEEPARstart{M}{obile} robots must be able to determine their location within their region of operation, a process known as localization, and navigate to other locations reliably in order to perform their tasks succesfully.

If there is no predefined map of the region the robot must be able to build up a map of using its' sensors and simultaneously localize itself within that map. This process is known as SLAM\hfill \vspace{\baselineskip}

In this project SLAM is demonstrated in a supplied Gazebo world and a custom designed world.

\section{Background / Formulation}
Background

\subsection{Slam}
Slam

\section{Model Configuration}
Config 

\section{Results}
Results


\section{Discussion}
Discussion

\section{Conclusion / Future work}
Conclusion


\end{document}